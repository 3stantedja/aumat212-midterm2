\documentclass[12pt,a4paper]{article}
\usepackage{problems}
%\usepackage{lg_jsylvest}

\begin{document}
    \begin{problem}
        Consider the function \(f(x,y) = \sin^{-1}(y-x)\).
        \begin{enumerate}[(a)]
            \item Determine the domain and range  of \(f\). Sketch the domain in the plane. Based on your sketch, is the domain open/closed/neither? bounded/unbounded?
            
            \emph{Hint:} recall from AUMAT 112  that the function \(g(t) \sin^{-1} t\) has the domain \(-1 \leq t \leq 1\) and range \(\frac{\pi}{2} \leq g(t) \leq \frac{\pi}{2}\).

            \item Describe the level curves of \(f\). Sketch the level curves \(f(x,y) = k\) for the values \(k = -\frac{\pi}{4}, \ 0, \ \frac{\pi}{4}\) on the same set of axes.
            
            \textit{Help, you won't let me use a calculator:} \(\sin{} \frac{\pi}{4} = \frac{\sqrt{2}}{2} \approx 0.7\).

            \item Based on your answer to part (b), what can you say about the vector \[\mathbf{u} = \frac{1}{\left| \nabla f \right|} \nabla f\] at various points in the domain of \(f\), \textbf{without actually calculating} \(\nabla f\)? Briefly explain your reasoning.
        \end{enumerate}
    \end{problem}
    \begin{answer}
        Answer here
    \end{answer}
\end{document}