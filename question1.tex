\documentclass[12pt,a4paper]{article}
\usepackage{problems}
%\usepackage{lg_jsylvest}

\title{AUMAT 212 Midterm 2 --- Question 1}
\date{\today}

\begin{document}
    \begin{problem}
        Consider the function \(f(x,y) = \sin^{-1}(y-x)\).
        \begin{enumerate}[(a)]
            \item Determine the domain and range  of \(f\). Sketch the domain in the plane. Based on your sketch, is the domain open/closed/neither? bounded/unbounded?
            
            \emph{Hint:} recall from AUMAT 112  that the function \(g(t) \sin^{-1} t\) has the domain \(-1 \leq t \leq 1\) and range \(-\frac{\pi}{2} \leq g(t) \leq \frac{\pi}{2}\).

            \item Describe the level curves of \(f\). Sketch the level curves \(f(x,y) = k\) for the values \(k = -\frac{\pi}{4}, \ 0, \ \frac{\pi}{4}\) on the same set of axes.
            
            \textit{Help, you won't let me use a calculator:} \(\sin{} \frac{\pi}{4} = \frac{\sqrt{2}}{2} \approx 0.7\).

            \item Based on your answer to part (b), what can you say about the vector \[\mathbf{u} = \frac{1}{\left| \nabla f \right|} \nabla f\] at various points in the domain of \(f\), \textbf{without actually calculating} \(\nabla f\)? Briefly explain your reasoning.
        \end{enumerate}
    \end{problem}
    \begin{answer}
        Function is \(f(x,y) = \sin^{-1}(y-x)\). Sketches are on a separate attachment.
        \begin{enumerate}[(a)]
            \item The range of \(f\) would be \[f \in \left[-\frac{\pi}{2}, \frac{\pi}{2} \right],\] and the domain of \(f\) would be:
            \begin{align*}
                x &\in [0,1] \\
                y &\in [0,1].
            \end{align*}

            Based on the sketch, the domain appears to be closed and bounded.

            \item The level curve would look like a series of linear plots of the form \(y = x + \sin{}k\), where \(k\) is an arbitrary constant. To derive:
            \begin{align*}
                \textbf{let} &&  f(x,y) &= k \\
                \textbf{then} && k &= \sin^{-1} (y - x) \\
                && \sin{}k &= y - x \\
                && \therefore \, y &= x + \sin{}k.
            \end{align*}

            At \(k = -\frac{\pi}{4}, \ 0, \ \frac{\pi}{4}\), the level curves are
            \begin{align*}
                y &= x - 0.7, \\
                y &= x, \\
                y &= x + 0.7.
            \end{align*}
            
            \item The vector \(\mathbf{u}\) will point toward a direction where \(x\) becomes more positive and \(y\) tends to be more negative, perpendicular to the level curves. This is because as \(x\) increases, the \(z\) value decreases towards \(-\pi/2\), and as such will overpower the \(y\) value.
        \end{enumerate}
    \end{answer}
\end{document}